\documentclass[12pt]{article}
\usepackage{times}

\usepackage{xspace}
\usepackage{url}
\usepackage{booktabs}

\clubpenalty 10000
\widowpenalty 10000
\def\topfraction{0.9}
\def\bottomfraction{0.9}
\def\textfraction{0.1}

% margins
%\topmargin 0truein
\setlength{\topmargin}{-0.5in}
\setlength{\textheight}{9in}
\setlength{\oddsidemargin}{0in}
\setlength{\evensidemargin}{0in}
\setlength{\textwidth}{6.5in}

\begin{document}

\thispagestyle{empty}

\begin{description}
\item[Type:] Weekly reading report  \hfill {\bf Date:} \today
\item[Paper:] [Copy the citation from the webpage.] Backus, Beeber, Best,
  Goldberg, Haibt, Herrick, Nelson, Sayre, Sheridan, Stern, Ziller, Hughes,
  and Nutt, ``The Fortran Automatic Coding System,'' Proceedings of the Western
  Joint Computer Conference, pp.\ 187--198, Los Angeles, CA, February, 1957.
\end{description}
[\emph{Use the section titles as shown below.}]

\paragraph{Summary:} [\emph{The summary should contain a short description of
  the problem, solution, and meaning of the paper. For example, for
  this paper you might write the following.}] \\
(Problem) In the 1950s, computer languages and architectures were in a very
primitive state. In particular, there were no high-level languages, only
assembly languages. (Solution) Backus and his colleagues introduced the first
programming language (Fortran) and its compiler for translating from this
high-level language to machine language.  (Meaning) Before these inventions,
computers could only be programmed using assembly code.  By inventing the
first high-level language and its compiler, this paper changed the landscape
of computing by offering a much better way for people to interact with
computers.

\paragraph{Strengths:} [\emph{One to three sentences on strengths of the paper.}]
The paper shows for the first time how to translate from high-level Fortran (a
language that uses procedures, loops, arrays, and scalar variables) to an
intermediate form, perform optimizations, and then translate to assembly
code. It invented control-flow graphs and the first redundancy elimination
optimizations.

\paragraph{Weaknesses:} [\emph{One to three sentences on weaknesses of the
  paper.}]
The paper's weaknesses are that it does not discuss or prove why or if it is
always possible to perform this translation.  The paper does not present any
experiments to explain the effectiveness of the approach.

\paragraph{Analysis I:} [\emph{Discuss in detail some interesting aspect of the
paper in detail.}] \\
This paper totally changed the field of computing.  In this era, hardware was
extremely expensive and only a few people had access or the ability to program
it.  This paper set the stage for dramatically improving the productivity of
programmers.  All high-level programming evolved from this starting point, and
this paper started the programming language implementation field (i.e.,
compilers, interpreters, etc.).  It is truly amazing how much compiler
technology was invented and predicted by this single paper, e.g., basic
blocks, frequency profiling, and high-level intermediate representations.

For example, this paper introduces the \emph{separation of concerns}
principle. Consider register allocation. The compiler assumes an infinite
temporary register set during the rest of compilation, which simplifies the
rest of the compiler.  It does not have to consider how optimizations interact
with register allocation early in compilation.  In their system, towards the
end of compilation, the compiler assigns the 3 registers of the 701 based on
execution frequency from profiling.  It computes basic block live ranges and
finds interferences on the fly.  This powerful abstraction is very useful for
ignoring resource constraints and simplifying other parts of the compiler.
There are some places where it however breaks down, such as embedded, VLIW,
EDGE, and other architectures which have more than one resource constraint.
For example, VLIW machines must carefully deal with both register allocation
and instruction scheduling in a fixed width instruction.  Which resource do
you do first? And how does it constrain the others?  This problem remains an
important one in the literature for many situations, and the only solutions,
of which I am aware, have been iteration~\cite{CRAIG, maher}, which is not
completely satisfying.

[Aside: Brasier et al.\ present a solution to register allocation and
scheduling~\cite{CRAIG}.  They iterate between scheduling, which may add
register pressure, and register allocation.  If the schedule does not require
any \emph{spilling} (spilling means a value comes from loads and stores
instead of a register access), they are done, otherwise, they constrain the
schedule to limit increases to live ranges and thus obtain a register
allocation with less or no spills.]

\paragraph{Analysis II:} [\emph{Discuss a second aspect of the paper.}]
This paper did not include any experimental results.  A very interesting
experiment would have been to assign $K/2$ problems to $N$ programmers, half
performed in assembly and half in Fortran, and then assigned an additional
$K/2$ problems and switch the programmers. Since each programmer does 1/2 in
each language, this controls a bit for individual variation.  Then, the
researchers could measure programming time (productivity) --- the time
to a correct solution in the two languages. The researchers could also measure
execution time, comparing how fast the resulting programs execute, to
determine the differences if any.  One should also use different problem
sizes, and compare if any differences are correlated to program size.  Writing
a short program that solves a small problem in assembly should be is easier
than writing a long program that solves a harder problem.  Other interesting
experiments would include how much each of the optimizations improved
performance, to show the impact of register allocation, common subexpression
elimination, etc.

\subsection*{Other Advice.}

\paragraph{Analysis Topics.}
The following are some suggested topics to write about, but do not feel
constrained by this list.
\begin{itemize}
\item Describe an experiment that would help explain/explore the
  results better.
\item How did it impact the field?
\item What questions remain open?
\item What experiments are missing?
\item How does it really relate the previous research?
\item Future directions.
\item Some examples for which it will or will not work.
\item Could a similar paper be published today?
\item Ideas or thoughts it provoked.
\item Other interesting commentary.
\end{itemize}

\paragraph{Hints on \LaTeX.}
To make this document into a PDF file perform the following Unix commands:
\begin{verbatim}
  pdflatex critique
  bibtex critique
  pdflatex critique
  pdflatex critique
\end{verbatim}

\noindent This sequence will produce a file called \verb|critique.pdf| and
some other auxiliary files.  You can learn more about \LaTeX on the web at
\url{http://www.latex-project.org} and/or buy a book on it for reference.  If
you prefer to another word processor, you may use it.

\paragraph{Writing well.}
Your critiques should be clear and grammatically correct.  I highly recommend
you read books on writing well. I like Joseph M. Williams and his books, and
in particular his book called ``Style: The Basics of Clarity and Grace''
\cite{Williams:05}.

Please use the active voice: For example, the following is clearer ``Brasier
et al.\ built a compiler.'' than ``A compiler was built.''  When writing about
computer systems, you should identify the subject and the time (i.e., when the
action takes place).  For example, likely subjects include the compiler, the
microarchitecture, the virtual machine, the memory manager, and the operating
system.  Potential times are: ahead-of-time or just-in-time dynamic
compilation, design time, and run time.

Another pet peeve of mine is to never use ``this'' as the subject of your
sentence.  For example, write: ``Brasier et al.\ built a compiler.  This
compiler iterated scheduling and register allocation to solve resource
constraint problems.'' Do not write: ``A compiler was built. This iterated
scheduling and register allocation to solve resource constraint problems.''
Qualify the ``this'' with a noun.  Usually ``this'' is referring back to
something in the previous sentence and you should make it explicit to which
part of the sentence it refers.

\bibliographystyle{abbrv}
\bibliography{critique}
\newpage
\paragraph{Grading Rubric:} Content is graded as follows: 1 point for summary;
1 strengths; 1 weakness; 2.5 analysis I, and 2.5 analsyis II. Put the
following grading
rubric at the end of your critique.\\
\begin{tabular}{lcr}
  Content & & /8 \\ 
  Grammar & & /1 \\
  Clarity & & /1 \\
  Bonus & & /1 \\ \hline
  Total & & /10 \\
\end{tabular}

\end{document}
