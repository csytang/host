%------------------------------------
% Dario Taraborelli
% Typesetting your academic CV in LaTeX
%
% URL: http://nitens.org/taraborelli/cvtex
% DISCLAIMER: This template is provided for free and without any guarantee 
% that it will correctly compile on your system if you have a non-standard  
% configuration.
% Some rights reserved: http://creativecommons.org/licenses/by-sa/3.0/
%------------------------------------

%!TEX TS-program = xelatex
%!TEX encoding = UTF-8 Unicode

\documentclass[10pt, a4paper]{article}
\usepackage{fontspec} 

% DOCUMENT LAYOUT
\usepackage{geometry} 
\geometry{a4paper, textwidth=5.5in, textheight=8.5in, marginparsep=7pt, marginparwidth=.6in}
\setlength\parindent{0in}

% FONTS
\usepackage[usenames,dvipsnames]{xcolor}
\usepackage{xunicode}
\usepackage{xltxtra}
\defaultfontfeatures{Mapping=tex-text}
%\setromanfont [Ligatures={Common}, Numbers={OldStyle}, Variant=01]{Linux Libertine O}
%\setmonofont[Scale=0.8]{Monaco}
%%% modified by Karol Kozioł for ShareLaTeX use
\setmainfont[
  Ligatures={Common}, Numbers={OldStyle}, Variant=01,
  BoldFont=LinLibertine_RB.otf,
  ItalicFont=LinLibertine_RI.otf,
  BoldItalicFont=LinLibertine_RBI.otf
]{LinLibertine_R.otf}
\setmonofont[Scale=0.8]{DejaVuSansMono.ttf}

% ---- CUSTOM COMMANDS
\chardef\&="E050
\newcommand{\html}[1]{\href{#1}{\scriptsize\textsc{[html]}}}
\newcommand{\pdf}[1]{\href{#1}{\scriptsize\textsc{[pdf]}}}
\newcommand{\doi}[1]{\href{#1}{\scriptsize\textsc{[doi]}}}
% ---- MARGIN YEARS
\usepackage{marginnote}
\newcommand{\amper{}}{\chardef\amper="E0BD }
\newcommand{\years}[1]{\marginnote{\scriptsize #1}}
\renewcommand*{\raggedleftmarginnote}{}
\setlength{\marginparsep}{7pt}
\reversemarginpar

% HEADINGS
\usepackage{sectsty} 
\usepackage[normalem]{ulem} 
\sectionfont{\mdseries\upshape\Large}
\subsectionfont{\mdseries\scshape\normalsize} 
\subsubsectionfont{\mdseries\upshape\large} 

% PDF SETUP
% ---- FILL IN HERE THE DOC TITLE AND AUTHOR
\usepackage[%dvipdfm, 
bookmarks, colorlinks, breaklinks, 
% ---- FILL IN HERE THE TITLE AND AUTHOR
	pdftitle={Yutian Tang - vita},
	pdfauthor={Yutian Tang},
	pdfproducer={http://nitens.org/taraborelli/cvtex}
]{hyperref}  
\hypersetup{linkcolor=blue,citecolor=blue,filecolor=black,urlcolor=MidnightBlue} 

% DOCUMENT
\begin{document}
{\LARGE Yutian	Tang}\\[1cm]
 Department of Computing\\
The Hong Kong Polytechnic University\\
Hong Kong \\[.2cm]
email: \href{mailto:csytang@comp.polyu.edu.hk}{csytang(AT)comp.polyu.edu.hk}\\
\textsc{url}: \href{http://www.chrisyttang.org}{http://www.chrisyttang.org}


%%\hrule
\section*{Current position}
\emph{Ph.D. candidate},The Hong Kong Polytechnic University, Hong Kong

%%\hrule
\section*{Areas of specialization}
 Software Product Line • Configuration System

%%\hrule
%\section*{Appointments held}
%\noindent
%\years{1903-1908}Swiss Patent Office, Bern\\
%\years{1908-1911}University of Bern\\
%\years{1911-1912}University of Zürich\\
%\years{1912-1914}Charles University of Prague\\
%\years{1914-1932}Prussian Academy of Sciences, Berlin\\
%\years{1920-1930}University of Leiden\\
%\years{1932-1955}Institute for Advanced Study, Princeton

%\hrule
\section*{Education}
\noindent
\years{2013}\textsc{BSc} in Computer Science, Jilin University, China\\
\years{2017}\textsc{PhD} in Software Engineering, The Hong Kong Polytechnic University, Hong Kong

%\hrule
\section*{Grants, honors, membership \& awards}
\noindent
\years{2017}Member of ISACA\\
\years{2016}Student member of IEEE\\
\years{2014}Student member of Hong Kong Computer Society\\
\years{2013-2017}Ph.D scholarship, Department of Computing, The Hong Kong Polytechnic University\\
\years{2012}Jilin University Scholarship, Third Price\\
\years{2012}Intermediate Title of Software Engineer(China Qualification Certificate of CS Tech. Proficiency)\\
\years{2011}Jilin University Scholarship, Second Price\\
\years{2011}Outstanding Student Award in College, Jilin University

\section*{Publications}

\noindent
\years{2017a}\textbf{Yutian Tang}, Hareton Leung, ``StiCProb: A Novel Feature Mining Approach Using Conditional Probability'', \emph{In Proceedings of 24th IEEE Internaltional Conference on Software Analysis, Evolution, and Reengineering (SANER)} pp 45-55.\\
\years{2017b}\textbf{Yutian Tang}, Hareton Leung, ``Constructing Feature Model by Identifying Variability-aware Modules'', \emph{In Proceedings of 25th IEEE International Conference on Program Comprehension (ICPC) pp 263-274}.\\
\years{2015a}\textbf{Yutian Tang}, Hareton Leung, ``A Top-down Feature Mining Framework for Software Product Line''. \emph{In Proceedings of International Conference on Enterprise Information System (ICEIS)} pp 71-81.\\
\years{2015b}\textbf{Yutian Tang}, Hareton Leung, ``Feature Mining for Product Line Construction'', \emph{The First International Conference on Advances and Trends in Software Engineering(SOFTENG)} pp29-33. 


%\vfill
\section*{Projects \& Tools}

\years{2017-current}\textbf{TypeC: Variability-aware C-program IDE}. [Java][Eclipse-plugin][Compiler-tool]
\begin{enumerate}
	\item a program checker for type checking, syntax checking and extract variability information from C program; it is developed as an Eclipse plugin;
	\item \href{http://www.chrisyttang.org/typec/}{http://www.chrisyttang.org/typec/} \\
\end{enumerate}


\years{2016-current}\textbf{LoongFMR: Loong Feature Model Recovery Toolkit}. [Java][Eclipse-plugin][Architecture]
\begin{enumerate}
	\item an architecture extractor and feature model verifier for software product line; it is developed as an Eclipse plugin;
	\item it also integrates with five related strategies, including NLP;
	\item \href{http://www.chrisyttang.org/loong\_fmr/}{http://www.chrisyttang.org/loong\_fmr/}
\end{enumerate}


\years{2015-current} \textbf{Loong: Colored IDE for Feature Locating} [Java][Eclipse-plugin][Code-2-Architecture]
\begin{enumerate}
	\item Loong is a software product line tool for analyzing and decomposing legacy code and constructing product line and it follows the paradigm of virtual separation of concerns;
	\item \href{http://www.chrisyttang.org/loong/}{http://www.chrisyttang.org/loong/}
\end{enumerate}

\section*{Employment}
\years{11/2013-8/2014} \textbf{Part-time Student Helper@The Hong Kong Polytechnic University, Hong Kong}
	\begin{enumerate}
		\item support website design and provide technical support in backend development for Education Development Center.
	\end{enumerate}

\section*{Teaching}

I service as a Teaching Assistant for following subjects.


\years{2016/17a} \textbf{COMP3235: Software Project Management}

\years{2016/17b} \textbf{COMP3233: Software Testing and Quality Assurance}

\years{2015/16a} \textbf{COMP4911: Capstone Project}

\years{2015/16b} \textbf{COMP3211: Software Engineering}

\years{2014/15} \textbf{COMP3235: Software Project Management}

\years{2013/14a} \textbf{COMP309: System Programming}

\years{2013/14b} \textbf{COMP322: Enterprise Information Systems Project Implementation}


\section*{Skill}
\subsection{Languages} Java, C, C++, Python, HTML, CSS, Javascript, Php, MySQL
\subsection{Tools} Soot, Bootstrap, Eclipse JDT, CDT, Matlab


%\hrule
%\section*{Service to the profession}

%\vspace{1cm}
\vfill{}
%\hrulefill

\begin{center}
{\scriptsize  Last updated: \today\- •\- 
% ---- PLEASE LEAVE THIS BACKLINK FOR ATTRIBUTION AS PER CC-LICENSE
Typeset in \href{http://nitens.org/taraborelli/cvtex}{
%\fontspec{Times New Roman}
\XeTeX }\\
% ---- FILL IN THE FULL URL TO YOUR CV HERE
\href{http://www.chrisyttang.org/cv.pdf}{http://www.chrisyttang.org/cv.pdf}}
\end{center}

\end{document}